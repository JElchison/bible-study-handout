\documentclass{bible-study-handout}

\begin{document}
\begin{figure*}[t]
  \centerline{\includegraphics{03-04_Exodus_art_FNL.jpg}}
\end{figure*}

\begin{bookoutline}
# Exodus of Israel from Egypt (1:1–18:27)
## Setting: Israel in Egypt (1:1–2:25)
## Call of Moses (3:1–4:31)
## Moses and Aaron: initial request (5:1–7:7)
## Plagues and exodus (7:8–15:21)
## Journey (15:22–18:27)
# Covenant at Sinai (19:1–40:38)
\end{bookoutline}

\begin{studyblock*}\textbf{Ex. 19:1–40:38} Covenant at Sinai. The second half of Exodus focuses on the events at Mount Sinai and the content of the law revealed to Moses. The narrative includes: the preparation of the people (19:1–25); the Ten Commandments and other laws (20:1–23:19); the instructions for entering the land (23:20–33); the confirmation of the covenant (24:1–18); the instructions relating to the tabernacle (25:1–31:18; 35:1–3); the breach, intercession, and renewal of the covenant (32:1–34:35); and the assembling of the tabernacle (35:4–40:38).\end{studyblock*}

\begin{bookoutline}
## Setting: Sinai (19:1–25)
## Covenant words and rules (20:1–23:33)
\end{bookoutline}

\begin{studyblock*}\textbf{Ex. 20:1–23:33} Covenant Words and Rules. This section records what will later be referred to as the Book of the Covenant (24:7) and includes: the Ten Commandments (20:1–21); instructions on worship (20:22–26; 23:10–19); rules and principles for community life (21:1–23:9); and instructions for entering the land of Canaan (23:20–33).\end{studyblock*}

\begin{bookoutline}
### The Ten Commandments (20:1–21)
\end{bookoutline}

\begin{studyblock*}\textbf{Ex. 20:1–21} The Ten Commandments. (See note on Deut. 5:1–21.) The Ten Commandments or “ten words” (see esv footnote at Ex. 34:28) are highlighted as the core of the covenant stipulations revealed to Moses; they define the life that the Lord calls his people to live before him (20:1–11) and with each other (vv. 12–17). The commandments are not exhaustive even in the areas to which they relate, but indicate to Israel how to remain faithful to the Lord. After Israel has wandered in the wilderness for forty years, Moses will restate the commandments to the generation that is about to enter the land of Canaan (see Deut. 5:6–21 and notes). NT authors assume the applicability of these commands in shaping the moral life of both Jewish and Gentile Christians (e.g., Rom. 13:9–10; Eph. 6:2).\end{studyblock*}

\newpage
\section{Exodus 20}

\subsection{The Ten Commandments}

\begin{studycomment}\textbf{Ex. 20:1} When \textbf{God spoke all these words}, he did so in such a way that all the people could hear. Cf. the repeated description of the sights and sounds of the Lord’s presence on Mount Sinai (19:16–20; 20:18); and “I have talked with you from heaven” (v. 22).\end{studycomment}\V{1}And God spoke all these words, saying,

\begin{studycomment}\textbf{Ex. 20:2} \textbf{I am the \textsc{Lord}} \textbf{your God, who brought you out of … Egypt}. As a preface to the Ten Commandments and the rest of the law, this description signifies that Israel’s call to covenant faithfulness is preceded by and based upon the Lord’s acts on their behalf in covenant relationship. Israel’s obedience to the commandments is the means by which they are to appropriate and enjoy what the Lord has already done by delivering them from Egypt and taking them to be his possession. The Lord will use the deliverance from Egypt to identify himself throughout Israel’s history, often to call them to remember what he has done for them and to live accordingly (e.g., Judg. 6:8; 1 Sam. 10:18; Ps. 81:10; Jer. 34:13).\end{studycomment}\V{2}“I am the \textsc{Lord} your God, who brought you out of the land of Egypt, out of the house of slavery.

\begin{studycomment}\textbf{Ex. 20:3} \textbf{You shall have no other gods}. Yahweh demands exclusive covenant loyalty. As the one true God of heaven and earth, Yahweh cannot and will not tolerate the worship of any “other gods” (cf. 22:20; 23:13, 24, 32); in other words, monotheism, the worship of the one true God, is the only acceptable belief and practice. \textbf{before me}. This Hebrew expression has been taken to mean “in preference to me,” or “in my presence,” or “in competition with me.” Most likely, “in my presence” (i.e., worshiping other gods in addition to the Lord) is the intended sense here, in view of (1) the creation account (Gen. 1:1–2:3), which makes any “other gods” irrelevant (since only the Lord is active); (2) the events in Egypt, in which the Lord displayed his superiority to “other gods” (cf. Ex. 12:12; 15:11; Ezek. 20:7–8); and (3) the persistent call to worship Yahweh alone (Ex. 22:20; 23:13, 24, 32–33; cf. Deut. 6:13–15). Even though this commandment does not comment on whether these “other gods” might have some real existence, Moses’ statement to a later generation makes clear that only “the \textsc{Lord} is God; there is no other besides him” (Deut. 4:35, 39; see also Ps. 86:10; Isa. 44:6, 8; 45:5, 6, 18; and 1 Cor. 8:4–6). See also note on Deut. 5:7.\end{studycomment}\V{3}“You shall have no other gods before me.

\begin{studycommentinline}\textbf{Ex. 20:4–6} \textbf{You shall not make for yourself a carved image}. The gods of both Egypt and Canaan were often associated with some aspect of creation and worshiped as, or through, an object that represented them. The Lord has made it clear, through the plagues and the exodus, that he has power over every aspect of creation because the whole earth is his (9:29; 19:5), and thus he commands Israel to refrain from crafting an image of anything in \textbf{heaven} or \textbf{earth} for worship (20:4–5a). The prohibition is grounded in the fact that the Lord is a jealous God (see 34:14; Deut. 6:15), and that the Lord has no physical form, and should not be thought to be localized in one (Deut. 4:15–20). Israel saw what happened to Egypt when Pharaoh refused to acknowledge what was being revealed about the Lord; here Israel is warned against doing the same, while also being reassured that their God is merciful and gracious (see Ex. 34:6–7).\end{studycommentinline}

\V{4}“You shall not make for yourself a carved image, or any likeness of anything that is in heaven above, or that is in the earth beneath, or that is in the water under the earth.

\begin{studycommentinline}\textbf{Ex. 20:5–6} \textbf{a jealous God}. God the Creator is worthy of all honor from his creation. Indeed, his creatures (mankind esp.) are functioning properly only when they give God the honor and worship that he deserves. God’s jealousy is therefore also his zeal for his creatures’ well-being. \textbf{visiting the iniquity of the fathers on the children}. Human experience confirms that immoral behavior on the part of parents often results in suffering for their children and grandchildren. This is one of the grievous aspects of sin, that it harms others besides the sinner himself. But this general principle is qualified in two ways: First, it applies only to \textbf{those who hate me}, i.e., to those who persist in unbelief as enemies of God. The cycle of sin and suffering can be broken through repentance. Second, the suffering comes to \textbf{the third and the fourth generation}, while God shows \textbf{steadfast love} (v. 6) to another group of people, namely, to \textbf{thousands of those who love me and keep my commandments} (i.e., to the thousandth generation; see esv footnote, and cf. Deut. 7:9).\end{studycommentinline}

\noindent{}\V{5}You shall not bow down to them or serve them, for I the \textsc{Lord} your God am a jealous God, visiting the iniquity of the fathers on the children to the third and the fourth generation of those who hate me, \V{6}but showing steadfast love to thousands of those who love me and keep my commandments.

\begin{studycomment}\textbf{Ex. 20:7} Taking the Lord’s name \textbf{in vain} (see note on Deut. 5:11) refers primarily to someone taking a deceptive oath in God’s name or invoking God’s name to sanction an act in which the person is being dishonest (Lev. 19:12). It also bans using God’s name in magic, or irreverently, or disrespectfully (Lev. 24:10–16). The Lord revealed his name to Moses (Ex. 3:14–15), and he has continued to identify himself in connection with his acts on Israel’s behalf (see 6:2, 6–8). Yahweh is warning Israel against using his name as if it were disconnected from his person, presence, and power.\end{studycomment}\V{7}“You shall not take the name of the \textsc{Lord} your God in vain, for the \textsc{Lord} will not hold him guiltless who takes his name in vain.

\begin{studycommentinline}\textbf{Ex. 20:8–11} Israel is to \textbf{remember the Sabbath day} by keeping it \textbf{holy} (v. 8; see notes on Deut. 5:12–15). The Lord had already begun to form the people’s life in the rhythm of working for \textbf{six days} (Ex. 20:9) and resting on the \textbf{seventh day} as a \textbf{Sabbath} (v. 10) through the instructions for collecting manna (see 16:22–26). Here the command is grounded further in the way that it imitates the Lord’s pattern in creation (20:11; see Gen. 2:1–3). Every aspect of Israel’s life is to reflect that the people belong to the Lord and are sustained by his hand. The weekly pattern of work and rest is to be a regular and essential part of this (see Ex. 31:12–18). In Deut. 5:15, Moses gives another reason for observing the day: it recalls their redemption from slavery in Egypt.\end{studycommentinline}

\V{8}“Remember the Sabbath day, to keep it holy. \V{9}Six days you shall labor, and do all your work, \V{10}but the seventh day is a Sabbath to the \textsc{Lord} your God. On it you shall not do any work, you, or your son, or your daughter, your male servant, or your female servant, or your livestock, or the sojourner who is within your gates. \V{11}For in six days the \textsc{Lord} made heaven and earth, the sea, and all that is in them, and rested on the seventh day. Therefore the \textsc{Lord} blessed the Sabbath day and made it holy.

\begin{studycomment}\textbf{Ex. 20:12} \textbf{Honor your father and your mother}. The word “honor” means to treat someone with the proper respect due to the person and their role. With regard to parents, this means (1) treating them with deference (cf. 21:15, 17); (2) providing for them and looking after them in their old age (for this sense of honor, see Prov. 3:9). Both Jesus and Paul underline the importance of this command (Mark 7:1–13; Eph. 6:1–3; 1 Tim. 5:4). This is the only one of the Ten Commandments with a specific promise attached to it: \textbf{that your days may be long}—meaning not just a long life, but one that is filled with God’s presence and favor. See note on Deut. 5:16.\end{studycomment}\V{12}“Honor your father and your mother, that your days may be long in the land that the \textsc{Lord} your God is giving you.

\newpage
\begin{studycomment*}\textbf{Ex. 20:13–15} The sixth through eighth commandments present general prohibitions not to \textbf{murder} (v. 13; see note on Deut. 5:17), \textbf{commit adultery} (v. 14), or \textbf{steal} (v. 15). In doing so, they set minimum standards for Israel to be a just society and indicate the context in which the people will be called further to be holy and to love the Lord with all their heart, soul, and might (Deut. 6:4–9), and their neighbors with goodwill and generosity (Lev. 19:18). Thus, while the prohibition against stealing is a basic principle of justice in Israel’s national life, the people are called to do more than refrain from taking another person’s possessions. They are to embody the Lord’s love for them by loving the stranger and sojourner as themselves (Lev. 19:33–34). When Jesus refers to the law in the Sermon on the Mount (“you have heard that it was said,” Matt. 5:21ff.), he is correcting not the intended purpose of the OT law but the mistaken presumption that these laws (or their interpretation) were meant to be exhaustive of what it meant to live as a child of the kingdom of heaven. (E.g., as Jesus made clear, simply refraining from murder does not fulfill the law when a person disdains his brother as a fool; or simply refraining from adultery does not fulfill the law when a man lusts after a woman; see Matt. 5:21–24, 27–28; and note on Matt. 5:21–48.)\end{studycomment*}

\V{13}“You shall not murder.

\V{14}“You shall not commit adultery.

\V{15}“You shall not steal.

\begin{studycomment}\textbf{Ex. 20:16} Acting as a \textbf{false witness} (see 23:1–3) suggests a legal trial in which false testimony could lead to punishment for one’s \textbf{neighbor}. Bearing “false witness” is condemned in Scripture for its disastrous effects among people and its utter disregard for God’s character (see Prov. 6:16–19; 12:22; 19:5, 9). The Lord’s righteousness and justice were to be reflected in Israel’s life as a nation, which was thus to exclude speaking falsely, especially for the sake of gaining something at the expense of another person and perverting justice.\end{studycomment}\V{16}“You shall not bear false witness against your neighbor.

\begin{studycomment}\textbf{Ex. 20:17} While the previous four commandments focus on actions committed or words spoken (vv. 13–16), the tenth commandment warns against allowing the heart to \textbf{covet … anything that is your neighbor’s}. When a person covets, he allows the desire for that which is coveted to govern his relationship with other people; this may become the motivation for murder, stealing, or lying either to attain the desired thing or to keep it from someone else. Because of the way that coveting values a particular thing over trust in and obedience to the Lord as the provider, it is also a breach of the first commandment, which the apostle Paul makes clear when he refers to coveting as idolatry (Eph. 5:5; Col. 3:5).\end{studycomment}\V{17}“You shall not covet your neighbor’s house; you shall not covet your neighbor’s wife, or his male servant, or his female servant, or his ox, or his donkey, or anything that is your neighbor’s.”

\begin{studycommentinline}\textbf{Ex. 20:18–20} The last time Israel had experienced a sign of \textbf{thunder} and \textbf{lightning}, it was in the context of the plague of hail sent on Egypt (see 9:23–26). Moses tells the people not to \textbf{fear} that God would kill them (20:20), explaining that God is testing them so that their life in the land might be governed by the \textbf{fear} of the Lord (see Deut. 6:2).\end{studycommentinline}

\V{18}Now when all the people saw the thunder and the flashes of lightning and the sound of the trumpet and the mountain smoking, the people were afraid and trembled, and they stood far off \V{19}and said to Moses, “You speak to us, and we will listen; but do not let God speak to us, lest we die.” \V{20}Moses said to the people, “Do not fear, for God has come to test you, that the fear of him may be before you, that you may not sin.” \V{21}The people stood far off, while Moses drew near to the thick darkness where God was.

\begin{bookoutline}
### Worship instructions: against idols and for an altar (20:22–26)
\end{bookoutline}

\begin{studycomment*}\textbf{Ex. 20:22–26} Worship Instructions: Against Idols and for an Altar. Together with 23:10–19, these verses frame the first section of laws following the Ten Commandments (21:1–23:9) and focus on Israel’s worship. Israel’s relationship with the Lord is her first priority (see the sequence of the Ten Commandments). This is reflected again here in that these religious regulations precede those on relating to one’s neighbor. These rules give more detailed explanations of the obligations implied by the first and second commandments (20:3–6).\end{studycomment*}

\subsection{Laws About Altars}

\V{22}And the \textsc{Lord} said to Moses, “Thus you shall say to the people of Israel: ‘You have seen for yourselves that I have talked with you from heaven. \V{23}You shall not make gods of silver to be with me, nor shall you make for yourselves gods of gold. \V{24}An altar of earth you shall make for me and sacrifice on it your burnt offerings and your peace offerings, your sheep and your oxen. In every place where I cause my name to be remembered I will come to you and bless you. \V{25}If you make me an altar of stone, you shall not build it of hewn stones, for if you wield your tool on it you profane it. \V{26}And you shall not go up by steps to my altar, that your nakedness be not exposed on it.’

\begin{bookoutline}
### Detailed legislation (21:1–23:19)
### Commands for the conquest (23:20–33)
## Covenant confirmed (24:1–18)
## Instructions for the tabernacle (25:1–31:17)
## Moses receives the tablets (31:18)
## Covenant breach, intercession, and renewal (32:1–34:35)
## Tabernacle: preparation for the presence (35:1–40:38)
\end{bookoutline}

\begin{fullwidth}
\bigskip\scriptsize\textit{Scripture quotations are from The Holy Bible, English Standard Version®, copyright © 2001 by Crossway Bibles, a publishing ministry of Good News Publishers. Used by permission. All rights reserved. Study notes are from ESV Study Bible, copyright © 2001 by Crossway Bibles.}
\end{fullwidth}

\end{document}
